%%
%% Copyright (c) 2018-2019 Weitian LI <wt@liwt.net>
%% CC BY 4.0 License
%%
%% Created: 2018-04-11
%%

% Chinese version
\documentclass[zh]{resume}

% Adjust icon size (default: same size as the text)
\iconsize{\Large}

% File information shown at the footer of the last page
\fileinfo{%
  % \faCopyright{} 2018--2020, Weitian LI \hspace{0.5em}
  % \creativecommons{by}{4.0} \hspace{0.5em}
  % \githublink{liweitianux}{resume} \hspace{0.5em}
  % \faEdit{} \today
}

\name{李华康}{}

% \keywords{BSD, Linux, Programming, Python, C, Shell, DevOps, SysAdmin}

% \tagline{\icon{\faBinoculars}} <position-to-look-for>}
% \tagline{<current-position>}

% \photo{<height>}{<filename>}

\profile{
  \mobile{155-2717-8316}
  \email{lihk@whu.edu.cn}
  \github{li-huakang} \\
  \university{中国科学院大学}
  \degree{电力电子 \textbullet 硕士}
  \birthday{1997-11-01}
  \address{北京}
  % Custom information:
  % \icontext{<icon>}{<text>}
  % \iconlink{<icon>}{<link>}{<text>}
}

\begin{document}
\makeheader

%======================================================================
% Summary & Objectives
%======================================================================
{\onehalfspacing\hspace{0em}%
武汉大学电气专业本科,保研至中科院电工研究所攻读电力电子专业硕士。有扎实的电气和电力电子专业知识基础。\\
硕士课题研究方向为电力半导体器件与模块相关,
参与电力电子变流器开发项目。\\
对半导体物理与半导体产业有浓厚兴趣。
熟悉计算机科学,有一定的软硬件开发经验。
学习能力强。\\
希望从事半导体芯片后端测试相关工作。
\par}

%======================================================================
\sectionTitle{教育背景}{\faGraduationCap}
%======================================================================
\begin{educations}
  \education%
    {2019.09}%
    % [2013.06]%
    {中国科学院大学}%
    {电力电子与电力传动}%
    {工学硕士}%
    {GPA:3.67/4.0}
    

  \separator{0.1ex}
  \education%
    {2015.09}%
    [2019.06]%
    {武汉大学}%
    {电气工程及其自动化}%
    {工学学士}%
    {GPA:3.69/4.0 (排名:12\%)}
\end{educations}

%======================================================================
\sectionTitle{知识体系与技能}{\faWrench}
%======================================================================
\begin{competences}
  \comptence{专业知识}{%
  电路、信号与系统、数字信号处理、模电、数电、微机技术、自动控制理论、
    半导体器件物理学。
  }
  \comptence{专业技能}{
  电路设计:LTspice, Kicad, Altium Designer;多物理场仿真:COMSOL。
  }
  \comptence{计算机相关}{%
  Java, Python, C编程;Linux, SSH, Git, Shell工具;\LaTeX , Matplotlib排版作图。
  }
  \comptence{\icon{\faLanguage} 语言}{
  \textbf{英语} --- 六级通过,读写(优良),听说(日常交流)\quad \textbf{日语} --- N3通过,简单阅读
  }

    
\end{competences}

%======================================================================
\sectionTitle{个人项目和论文}{\faCode}
%======================================================================
\begin{itemize}
  \item 技术知识文档:\link{https://github.com/Li-Huakang/SVPWM}{\texttt{SVPWM数学原理与实现
  }}、
  \link{https://github.com/Li-Huakang/SCDP-Notes}{\texttt{半导体器件物理笔记}};课程设计:电机模型仿真、电力电子仿真等。
  \item 有丰富的Python脚本程序编写经验:文件操作、数据处理、绘图等。
  \item 第一作者论文:{\small A Combination Method for Full-time and Comprehensive Virtual Junction Temperature Estimation of IGBT Power Module, The 34th International Electric Vehicle Symposium \& Exhibition, 2021.6. (国际会议)}
  \item 其他作者论文:{\small An Electrothermal Model for IGBT Based on Finite Differential Method, IEEE Journal of Emerging and Selected Topics in Power Electronics, 2020.03. 
  (SCI期刊) }
  \item 专利:一种驱动装置,基于ESP32芯片,特点为支持通过网页进行无线控制和精确控制驱动脉冲宽度和时序。
\end{itemize}

%======================================================================
\sectionTitle{科研项目}{\faAtom}
%======================================================================
\begin{itemize}
  \item \textbf{1200V/600A SiC HPD 、650V/400A HP1 、900V/600A SiC等功率半导体模块开发与测试}\\
  {\small 基于课题组功率半导体封装实验室,参与的主要工作有:1.电气特性测试与评估,动静态特性及其在特定工况下的性能;
  2.热特性测试与评估,稳态、瞬态热阻测试,模块工作热稳定性研究;
  3.参与模块设计封装。}
  \item \textbf{功率半导体模块健康状态管理研究}:可靠性、寿命预测\\ 
  {\small 1.调研功率半导体模块失效机理和主要失效模式;
  2.使用先兆参数反映模块产品的退化特性;
  3.讨论基于数据驱动和故障物理的健康状态监测方案。}
  \item 研究生课题《\textbf{基于多参数测量的功率半导体结温监测方法}》\\
  {\small 基于某550kW电机驱动系统中IGBT功率模块的在线结温监测项目。
  基本原理是将功率半导体器件本身当作温度传感器,对器件进行实时的结温监测。
  主要工作有方案的提出、实施与验证,电路板的开发与调试。
  研究生论文在此基础上引入模型参考自适应,提出利用卡尔曼滤波器进行信息融合,扩展结温检测方法的应用工况范围和准确度。}
  \item 本科毕业设计论文《\textbf{逆变器电磁干扰(EMI)抑制研究}》\\ 
  {\small 主要研究电力电子逆变器电磁干扰的产生、传播和抑制:
  1.通过仿真和实验的方法分析无源器件的高频特性;
  2.总结电力电子EMI滤波器设计方法,对非理性情况下的滤波器性能进行分析;
  3.利用随机开关频率PWM控制方法抑制EMI产生。}
  
\end{itemize}

%======================================================================
\sectionTitle{获奖}{\faAward}
%======================================================================
\begin{itemize}
  \item 全国大学生数学竞赛二等奖
  \item 武汉大学三好学生、武汉大学曾宪梓奖学金、武汉大学暑期社会实践一等奖
  \item 中国科学院大学学业奖学金
\end{itemize}


\end{document}
