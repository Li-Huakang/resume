%%
%% Copyright (c) 2018-2019 Weitian LI <wt@liwt.net>
%% CC BY 4.0 License
%%
%% Created: 2018-04-11
%%

% Chinese version
\documentclass[zh]{resume}

% Adjust icon size (default: same size as the text)
\iconsize{\Large}

% File information shown at the footer of the last page
\fileinfo{%
  % \faCopyright{} 2018--2020, Weitian LI \hspace{0.5em}
  % \creativecommons{by}{4.0} \hspace{0.5em}
  % \githublink{liweitianux}{resume} \hspace{0.5em}
  % \faEdit{} \today
}

\name{李华康}{}

% \keywords{BSD, Linux, Programming, Python, C, Shell, DevOps, SysAdmin}

% \tagline{\icon{\faBinoculars}} <position-to-look-for>}
% \tagline{<current-position>}

% \photo{<height>}{<filename>}

\profile{
  \mobile{155-2717-8316}
  \email{lihk@whu.edu.cn}
  \github{li-huakang} \\
  \university{中国科学院大学}
  \degree{电力电子 \textbullet 硕士}
  \birthday{1997-11-01}
  \address{北京}
  % Custom information:
  % \icontext{<icon>}{<text>}
  % \iconlink{<icon>}{<link>}{<text>}
}

\begin{document}
\makeheader

%======================================================================
% Summary & Objectives
%======================================================================
{\onehalfspacing\hspace{2em}%
武汉大学电气专业本科,保研至中国科学院电工研究所攻读电力电子专业硕士。
有扎实的电气和电力电子专业知识基础,
擅长物理建模和数学分析。
硕士课题研究方向为功率半导体器件与模块相关,
熟悉半导体物理学,熟悉功率模块失效与退化机理,专攻器件结温监测。
熟悉计算机科学,有一定的硬件开发经验。
对新知识有良好的学习吸收能力。
希望从事电力电子技术研究、功率模块开发与测试的相关工作。
\par}

%======================================================================
\sectionTitle{教育背景}{\faGraduationCap}
%======================================================================
\begin{educations}
  \education%
    {2019.09}%
    % [2013.06]%
    {中国科学院大学}%
    {电力电子与电力传动}%
    {工学硕士}%
    {GPA:3.67/4.0}
    

  \separator{0.1ex}
  \education%
    {2015.09}%
    [2019.06]%
    {武汉大学}%
    {电气工程及其自动化}%
    {工学学士}%
    {GPA:3.69/4.0 (排名:12\%)}
\end{educations}

%======================================================================
\sectionTitle{知识体系与技能}{\faWrench}
%======================================================================
\begin{competences}
  \comptence{专业知识}{%
  现代电力电子技术(91)、电路原理(96)、电机学(97)、自动控制理论(97)、信号与系统(95)、高电压工程技术(91)、
    半导体器件物理学(选修)。
  }
  \comptence{专业技能}{
    Matlab/Simulink, Ansys Icepak, COMSOL, LTspice, Kicad, Altium Designer, SOLIDWORKS;Arduino、DSP等。
  }
  \comptence{计算机相关}{%
    Java, Python, C编程;Linux, SSH, Git, Shell工具;\LaTeX , Matplotlib排版作图。
  }
  \comptence{\icon{\faLanguage} 语言}{
    \textbf{英语} --- 六级通过,读写(优良),听说(日常交流)\quad \textbf{日语} --- N3通过,简单阅读
  }

    
\end{competences}

%======================================================================
\sectionTitle{个人项目和论文}{\faCode}
%======================================================================
\begin{itemize}
  \item 技术知识文档:\link{https://github.com/Li-Huakang/SVPWM}{\texttt{SVPWM数学原理与实现
  }}、
  \link{https://github.com/Li-Huakang/SCDP-Notes}{\texttt{半导体器件物理笔记}};课程设计:电机模型仿真、电力电子仿真等。
  \item 本科毕业设计论文《逆变器电磁干扰抑制研究》:无源滤波器插入损耗仿真、随机PWM。
  \item 第一作者论文:{\small A Combination Method for Full-time and Comprehensive Virtual Junction Temperature Estimation of IGBT Power Module, The 34th International Electric Vehicle Symposium \& Exhibition, 2021.6. (国际会议)}
  \item 其他作者论文:{\small An Electrothermal Model for IGBT Based on Finite Differential Method, IEEE Journal of Emerging and Selected Topics in Power Electronics, 2020.03. 
  (SCI期刊) }
  \item 专利:一种驱动装置,基于ESP32芯片,特点为支持通过网页进行无线控制和精确控制驱动脉冲宽度和时序。
\end{itemize}

%======================================================================
\sectionTitle{科研项目}{\faAtom}
%======================================================================
\begin{itemize}
  \item \textbf{550kW电机控制器功率模块结温在线监测}\\
        {\small 该项目的目的是在某500kW电机驱动系统中实现功率模块的在线结温监测。本人参与的主要工作有:
        1.通过对比灵敏度、准确度、在线测量难度等,论证利用导通压降进行结温在线监测的合理性和可行性;
        2.通过调研确定导通压降的测量电路,并进行测量电路板的开发与调试;
        3.结温标定与模型验证,确定大电流下温度标定方法,与小电流(mA)法、红外测量法和热阻计算法进行对比验证;
        4.利用突然停止法实现在线三相功率实验验证。}
  % \item 本科毕业设计论文《逆变器电磁干扰抑制研究》:无源滤波器插入损耗仿真、随机PWM。
  \item \textbf{1200V/600A SiC HPD 、650V/400A HP1 、900V/600A SiC等功率模块开发与测试}\\
  {\small 参与的主要工作有:1.电气特性测试与评估,主要通过静态测试和双脉冲测试评估模块的动静态特性,通过模拟特定工况条件测试其工作性能;
  2.热特性测试与评估,进行稳态、瞬态热阻测试,提出评估在特定散热系统下模块工作极限的方法,研究在散热上制约模块应用能力的因素;
  3.参与模块设计封装。}
  \item 研究生课题《\textbf{基于多参数测量的功率模块结温监测方法}》:TSEPs、损耗模型、传热模型、信息融合。
  \item \textbf{调研功率模块健康状态监测与退化模型}:失效物理、先兆参数、监测方案。
  % \item 550kW电机控制器功率模块结温在线监测:导通压降、物理模型与仿真、硬件平台搭建与验证。
  % \item 基于信息融合的功率半导体结温估计:TSEPs、损耗模型、传热模型、卡尔曼滤波器。
  % \item 基于有限差分法的IGBT热电耦合模型:双极性输运方程、SPICE、有限差分。
  
\end{itemize}

%======================================================================
\sectionTitle{获奖}{\faAward}
%======================================================================
\begin{itemize}
  \item 全国大学生数学竞赛二等奖
  \item 武汉大学三好学生、武汉大学曾宪梓奖学金、武汉大学暑期社会实践一等奖
  \item 中国科学院大学学业奖学金
\end{itemize}

%======================================================================
% \sectionTitle{实习经历}{\faBriefcase}
% %======================================================================
% \begin{experiences}
%   \experience%
%     [2018.04]%
%     {2018.08}%
%     {数据工程师 @ 上海领脉网络科技(初创公司)}%
%     [\begin{itemize}
%       \item 从 Amazon 网页搜索并挖取商品与广告信息
%         (Python, Requests, BeautifulSoup)
%       \item 配置 Airflow 服务器和数据库等基础设施,
%         定期从 Amazon 获取产品销售与广告投放等数据
%       \item 开发网站(Flask, jQuery),帮助客户优化 Amazon 广告投放
%     \end{itemize}]

%   \separator{0.5ex}
%   \experience%
%     [2013.07]%
%     {2013.09}%
%     {网站开发 @ 97 随访(初创公司)}%
%     [\begin{itemize}
%       \item 后端开发(Django),完成用户注册、数据存储和搜索等功能
%       \item 前端开发(jQuery, AJAX),对患者各项指标随时间的变化进行可视化
%     \end{itemize}]
% \end{experiences}

\end{document}
