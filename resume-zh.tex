%%
%% Copyright (c) 2018-2019 Weitian LI <wt@liwt.net>
%% CC BY 4.0 License
%%
%% Created: 2018-04-11
%%

% Chinese version
\documentclass[zh]{resume}

% Adjust icon size (default: same size as the text)
\iconsize{\Large}

% File information shown at the footer of the last page
\fileinfo{%
  % \faCopyright{} 2018--2020, Weitian LI \hspace{0.5em}
  % \creativecommons{by}{4.0} \hspace{0.5em}
  % \githublink{liweitianux}{resume} \hspace{0.5em}
  % \faEdit{} \today
}

\name{李华康}{}

% \keywords{BSD, Linux, Programming, Python, C, Shell, DevOps, SysAdmin}

% \tagline{\icon{\faBinoculars}} <position-to-look-for>}
% \tagline{<current-position>}

% \photo{3}{resume.png}

\profile{
  \mobile{155-2717-8316}
  \email{huakang.li@outlook.com}
  \github{li-huakang} \\
  \university{武汉大学\textbullet 中国科学院大学}
  \degree{电力电子/电气工程 \textbullet 硕士}
  \birthday{1997-11-01}
  \address{上海}
  % Custom information:
  % \icontext{<icon>}{<text>}
  % \iconlink{<icon>}{<link>}{<text>}
}

\begin{document}
\makeheader

%======================================================================
% Summary & Objectives
%======================================================================
{\onehalfspacing\hspace{2em}%
小米手机电源技术组3年工作经验,主要负责无线充电系统仿真、电荷泵仿真优化、新技术预研等,同时有有线、无线项目开发经验以及芯片验证经验。
武汉大学电气专业本科,保研至中国科学院电工研究所攻读电力电子专业硕士,从事碳化硅功率模块封装、测试、应用相关研究工作。
熟悉计算机知识,有软硬件开发经验。
\par}

%======================================================================
\sectionTitle{教育背景}{\faGraduationCap}
%======================================================================
\begin{educations}
  \education%
    {2019.09}%
    [2022.06]%
    {中国科学院大学}%
    {电力电子与电力传动}%
    {工学硕士}%
    {GPA:3.7/4.0}
    
  \separator{0.1ex}
  \education%
    {2015.09}%
    [2019.06]%
    {武汉大学}%
    {电气工程及其自动化}%
    {工学学士}%
    {GPA:3.7/4.0 (保研)}
\end{educations}

\sectionTitle{工作经历}{\faBriefcase}
%======================================================================
\begin{experiences}
  \experience%
    [2022.08]%
    {现在}%
    {硬件研发工程师 @ \textbf{小米集团}}%
    [\begin{itemize}
      \item 负责上海区域无线充电项目关键问题看护
      \item 从0到1建设无线充电系统仿真评估平台,负责系统仿真与落地指导
      \item 主导2个无线充电创新技术预研,负责方案提出、调试、验证与推广
      \item 负责4个校企合作项目的课题提出、合作高校选择、项目管理与成果验收
    \end{itemize}]

  \separator{0.5ex}
  \experience%
    [2019.09]%
    {2022.06}%
    {硕士研究生 @ \textbf{中科院电工所}}%
    [\begin{itemize}
      \item 参与设计功率模块的封装设计,对样品电气和热性能进行测试
      \item 负责在线结温监测方案的原理分析、电路设计与方案验证
      \item 负责功率模块健康状态管理的前期调研
    \end{itemize}]
\end{experiences}



%======================================================================
\sectionTitle{知识体系与技能}{\faWrench}
%======================================================================
\begin{competences}
  \comptence{工作技能}{%
  预研项目管理与开发、新技术探索、校企合作、量产项目看护、技术文档维护。
  }
  \comptence{专业知识}{%
  电路、电力电子、信号与系统、模电、数电、自动控制理论、半导体器件物理学、传热学等。
  }
  \comptence{专业技能}{
  电路仿真:LTspice、Matlab;磁场仿真:HFSS;热仿真:COMSOL;编程:Python、Java、C。
  }
  \comptence{\icon{\faLanguage} 语言}{
    \textbf{英语} --- 六级,读写(优良),听说(日常交流)\quad \textbf{日语} --- N3,简单阅读与交流
  }

    
\end{competences}


%======================================================================
\sectionTitle{其他项目}{\faCode}
%======================================================================
\begin{itemize}
  \item IGBT、SiC驱动设计
  \item 逆变器电磁干扰(EMI)抑制方法(无源滤波器、随机PWM)研究
  \item ESP32的音频播放器(电源、音频电路设计、C程序编写与调试)
  \item 技术知识文档:\link{https://github.com/Li-Huakang/SVPWM}{\texttt{SVPWM数学原理与实现
  }}、\link{https://github.com/Li-Huakang/SCDP-Notes}{\texttt{半导体器件物理笔记}};课程设计:电机模型仿真、电力电子仿真等。
\end{itemize}


%======================================================================
\sectionTitle{专利与论文}{\faAtom}
%======================================================================
\begin{itemize}
  \item {\small 一作@Xiaomi: A Wideband Tuning Method for Extended Operating Frequency Range of Wireless Power Receiver, WPTCE 2024, 2024.5}
  \item 一作@CAS:{\small A Combination Method for Full-time and Comprehensive Virtual Junction Temperature Estimation of IGBT Power Module, EVS 34, 2021.6. (国际会议)}\\ 
  {\small Thermal Evaluation of SiC MOSFET Power Modules with HPD Package and Double-sided Cooling Package, PCIM Asia 2022, 2022.10.}
  \item 模块驱动、电荷泵拓扑、无线充电系统相关专利共计10+。
\end{itemize}



%======================================================================
\sectionTitle{荣誉}{\faAward}
%======================================================================
\begin{itemize}
  \item 小米集团上海区域总经理特别贡献奖(团队)、科技创新奖、优秀案例
  \item 中国科学院大学三好学生、全国大学生数学竞赛二等奖、武汉大学三好学生、武汉大学曾宪梓奖学金
  
\end{itemize}

\end{document}
